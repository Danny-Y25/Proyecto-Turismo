%Declaramos la clase de Documento
\documentclass[a4paper,12pt]{article}

%Paquete de idioma y codificacion de caracteres
\usepackage[utf8]{inputenc}
\usepackage[T1]{fontenc}

%Paquetes de escritura matematica
\usepackage[amsmath,amsfonts,amssymb]
%Otros paquetes
%Datos para el titulo de nuestro documento
\title{Proyecto de Turismo \Gualaquiza-Tourist {}}

\author{Danny Gustavo Yunga Yunga}
\date{7 de Noviembre de 2020}

%Dimencions del Documento
\usepackage[left=3cm,right=3cm]{geometry}

\begin{document}

%Imprpimimos la pagina de titulo

\maketitle

El desarrollo de la aplicación se centra en ayudar a los turistas que visitan la ciudad de Gualaquiza y al no poseer conocimiento de los sitios turisticos y demas opciones, carecen de la movilidad para poder disfrutar todo lo que tiene que ofrecer Gualaquiza, por ende la aplicacion abarcara lo mas importante de Gualaquiza en cuanto a turismo se refiera. La aplicacion poseera las opciones de: 

Sitios turisticos: la cual nos dara la informacion y fotografias de los lugares recreativos para el turista.

Gastronomia: esta opcion nos informara acerca de los platos tipicos que posee Gualaquiza y una informacion adicional de como esta compuesto dicho plato y un poco de la historia.

Hoteles: esta opcion le sera de gran ayuda a los turistas, debido a que indica la informacion de los hoteles disponibles dentro de Gualaquiza, con la informacion de contacto y al direccion de la misma.



\end{document}
